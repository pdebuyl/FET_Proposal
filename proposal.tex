\providecommand{\classoptions}{keys}
\documentclass[noworkareas,deliverables,\classoptions]{euproposal}       % for writing
%\documentclass[submit,noworkareas,deliverables]{euproposal}        % for submission
%\documentclass[submit,public,noworkareas,deliverables]{euproposal} % for public version

\usepackage[utf8]{inputenc}

\usepackage{float}  % used to suppress floating of tables in Resources section.
\usetikzlibrary{calc,fit,positioning,shapes,arrows,snakes}

\addbibresource{kwarc.bib}
%%% institutions
\WAinstitution[id=PAR1,
        countryshort=BE,
        acronym=Partner1]
        {Partner One}

\WAinstitution[id=PAR2,
        countryshort=BE,
        acronym=Partner2]
        {Partner Two}

\WAinstitution[id=PAR3,
        countryshort=BE,
        acronym=Partner3]
        {Partner Three}

\WAperson[id=PAR1P1,
           personaltitle=Prof. ,
           birthdate=1 Jan. 2000,
           academictitle=Professor,
           affiliation=PAR1,
           department=Department for Research,
           privaddress=None of your business,
           privtel=that neither,
           email=p1@par1.com,
           workaddress={Campus, 1, City},
           worktel=+1 234 56 78,
           workfax=N/A
           ]
           {Person 1}

%%% Local Variables: 
%%% mode: latex
%%% TeX-master: "proposal"
%%% End: 
 % Some sections of the included files depend on this.
\usepackage{comments}
\usepackage{framed}


\begin{document}

\begin{proposal}[
  % These PM numbers (person months) are for the coordinator to help planning
  % Participants should not change these, but add PM numbers in the CVS in
  % the site descriptions at CVs/*.tex
  site=PAR1,
  site=PAR2,
  site=PAR3,
  botupPM, % we want to work via bottom up PM distribution,
  coordinator=PAR1P1,
  coordinatorsite=PAR1,
  acronym={MyFETProject},
  acrolong={MyFETProject},
  title=My FET research project,
  callname=Topic: FET-Open - Novel ideas for radically new technologies,
  callid=H2020-FETOPEN-2014-2015-RIA,
  keywords={science, research, ideas},
  instrument= Call: H2020-FETOPEN-2014-2015, %Call: H2020-EINFRA-2015-1, 3 Topic 9-2015
  challengeid = TODO,
  months=48,
  compactht]
\newcommand{\TheProject}{\pn}% \pn is defined automatically
\begin{abstract}
  \TheProject is to do research.

  All produced code and tools will be open source.
\end{abstract}
\ifsubmit\else\setcounter{tocdepth}{4}\fi
\tableofcontents

\TOWRITE{PAR1P1}{Larger table of participants}
\TOWRITE{PAR1P1}{Abstract in the first page?}
\TOWRITE{PAR1P1}{Centering}

\begin{draft}
\section*{Things to do \dots}
\TOWRITE{All}{Request from PAR1P1}
\subsection*{Things PAR1P1 asks as the consortium to do}
\begin{verbatim}
- [ ] Do things.
\end{verbatim}
\end{draft}


% ---------------------------------------------------------------------------
%  Section 1: Excellence
% ---------------------------------------------------------------------------

\section{S\&T Excellence}

Body of section ``Excellence''.

\subsection{Targeted breakthrough, Long term vision and Objectives}\label{sec:objectives}

\eucommentary{
  \begin{compactitem}
  \item Describe the targeted scientific breakthrough of the project.
  \item Describe how the targeted breakthrough of the project contributes to a
    long-term vision for new technologies.
  \item Describe the specific objectives for the project, which should be clear,
    measurable, realistic and achievable within the duration of the project.
  \end{compactitem}
}

\subsection{Relation to the work programme}\label{sec:relation-wp}

\eucommentary{
  \begin{compactitem}
  \item Indicate the work programme topic to which your proposal relates, and
    explain how your proposal addresses the specific challenge and scope of that
    topic, as set out in the work programme.
  \end{compactitem}
}

\subsection{Novelty, level of ambition and foundational character}\label{sec:progress}

\eucommentary{
  \begin{compactitem}
  \item Describe the advance your proposal would provide beyond the
    state-of-the-art, and to what extent the proposed work is ambitious, novel
    and of a foundational nature. Your answer could refer to the ground-breaking
    nature of the objectives, concepts involved, issues and problems to be
    addressed, and approaches and methods to be used.
  \end{compactitem}
}

\subsection{Research methods}\label{sec:methods}
\eucommentary{
  \begin{compactitem}
  \item Describe the overall research approach, the methodology and explain its
    relevance to the objectives.
  \item Where relevant, describe how sex and/or gender analysis is taken into
    account in the project's content.
  \end{compactitem}
}

\subsection{Interdisciplinary nature}\label{sec:interdisc}

\eucommentary{
  \begin{compactitem}
  \item Describe the research disciplines involved and the added value of the inter-disciplinarity.
  \end{compactitem}
}

%%% Local Variables:
%%% mode: latex
%%% TeX-master: "proposal"
%%% End:


\clearpage

% ---------------------------------------------------------------------------
%  Section 2: Impact
% ---------------------------------------------------------------------------

\section{Impact}
\input{impact.tex}

\clearpage

% ---------------------------------------------------------------------------
%  Section 3: Implementation
% ---------------------------------------------------------------------------

\section{Implementation}

\subsection{Project work plan}
\label{sec:wp}

\eucommentary{
Please provide the following:
\begin{compactitem}
\item brief presentation of the overall structure of the work plan;
\item timing of the different work packages and their components (Gantt chart or
  similar);
\item detailed work description, i.e.:
  \begin{compactitem}
  \item a description of each work package (table 3.1a);
  \item a list of work packages (table 3.1b);
  \item a list of major deliverables (table 3.1c);
  \end{compactitem}
\item graphical presentation of the components showing how they inter-relate (Pert
  chart or similar).
\end{compactitem}
%
Give full details. Base your account on the logical structure of the project and
the stages in which it is to be carried out. Include details of the resources to
be allocated to each work package. The number of work packages should be
proportionate to the scale and complexity of the project.\\
%
You should give enough detail in each work package to justify the proposed
resources to be allocated and also quantified information so that progress can
be monitored, including by the Commission.\\
%
You are advised to include a distinct work package on 'management' (see section
3.2) and to give due visibility in the work plan to 'dissemination and
exploitation' and 'communication activities', either with distinct tasks or
distinct work packages.\\
%
You will be required to include an updated (or confirmed) 'plan for the
dissemination and exploitation of results' in both the periodic and final
reports. (This does not apply to topics where a draft plan was not required.)
This should include a record of activities related to dissemination and
exploitation that have been undertaken and those still planned. A report of
completed and planned communication activities will also be required.\\
%
If your project is taking part in the Pilot on Open Research Data%
\footnote{%
  Certain actions under Horizon 2020 participate in the 'Pilot on Open Research
  Data in Horizon 2020'. All other actions can participate on a voluntary basis
  to this pilot. Further guidance is available in the H2020 Online Manual on the
  Participant Portal.},
you must include a 'data management plan' as a distinct deliverable within the
first 6 months of the project. A template for such a plan is given in the
guidelines on data management in the H2020 Online Manual. This deliverable will
evolve during the lifetime of the project in order to present the status of the
project's reflections on data management.\\
%
Definitions:
\begin{description}
\item[Work package] means a major sub-division of the proposed project.
\item[Deliverable] means a distinct output of the project, meaningful in terms of the project's overall
  objectives and constituted by a report, a document, a technical diagram, a software etc.
\item[Milestones] means control points in the project that help to chart progress. Milestones
    may correspond to the completion of a key deliverable, allowing the next phase of the
    work to begin. They may also be needed at intermediary points so that, if problems have
    arisen, corrective measures can be taken. A milestone may be a critical decision point in
    the project where, for example, the consortium must decide which of several technologies
    to adopt for further development.
\end{description}
%
Report on work progress is done primarily through the periodic and final reports. Deliverables
should complement these reports and should be kept to the minimum necessary.
}

\subsection{Management and risk assessment}

\eucommentary{
  \begin{compactitem}
  \item Describe the organisational structure and the decision-making (including a list of
  milestones (table 3.2a)) .
\item Describe any critical risks, relating to project implementation, that the stated project's
  objectives may not be achieved. Detail any risk mitigation measures. Please provide a
  table with critical risks identified and mitigating actions (table 3.2b).
  \end{compactitem}
}

\subsection{Consortium as a whole}

\eucommentary{The individual members of the consortium are described in a
  separate section 4. There is no need to repeat that information here.\\
%
  \begin{compactitem}
  \item Describe the consortium. How will it match the project's objectives? How
    do the members complement one another (and cover the value chain, where
    appropriate)? In what way does each of them contribute to the project? How
    will they be able to work effectively together?
  \item If applicable, describe how the project benefits from any industrial/SME
    involvement.
  \item Other countries: If one or more of the participants requesting EU
    funding is based in a country that is not automatically eligible for such
    funding (entities from Member States of the EU, from Associated Countries
    and from one of the countries in the exhaustive list included in General
    Annex A of the work programme are automatically eligible for EU funding),
    explain why the participation of the entity in question is essential to
    carrying out the project.
  \end{compactitem}
}

\subsection{Resources to be committed}

\eucommentary{
  Please make sure the information in this section matches the costs as stated
  in the budget table in section 3 of the administrative proposal forms, and the
  number of person/months, shown in the detailed work package descriptions.\\
%
  \begin{compactitem}
  \item a table showing number of person/months required (table 3.4a)
  \item a table showing 'other direct costs' (table 3.4b) for participants where
    those costs exceed 15\% of the personnel costs (according to the budget
    table in section 3 of the administrative proposal forms)
  \end{compactitem}
}


\gantttaskchart[draft,xscale=.33,yscale=.33,milestones]

\newpage
\subsubsection{Deliverables}\label{sec:deliverables}
\inputdelivs{9.3cm}

\newpage
\subsubsection{Milestones}\label{sec:milestones}
\eucommentary{Milestones means control points in the project that help to chart progress. Milestones may
correspond to the completion of a key deliverable, allowing the next phase of the work to begin.
They may also be needed at intermediary points so that, if problems have arisen, corrective
measures can be taken. A milestone may be a critical decision point in the project where, for
example, the consortium must decide which of several technologies to adopt for further
development.}

\begin{milestones}
  \milestone[id=start,month=12,
  verif={Completed all corresponding deliverables and reported the progress in the 2nd Project meeting report.}]
  {Starting the project}
  {This is the first milestone.}

  \milestone[id=mymile1,month=24,
  verif={Completed all corresponding deliverables and reported the progress in the 4th Project meeting report.}]
  {Another milestone}
  {By this milestone the project will be in good way.}

  \milestone[id=mymile2,month=36,
  verif={Completed all corresponding deliverables and reported the progress in the 6th Project meeting report.}]
  {Additional milestone}
  {By this milestone the project will be almost over.}

  \milestone[id=finalmile,month=48,
  verif={Completed all corresponding deliverables and reported the progress in the 8th Project meeting report.}]
  {Final milestone}
  {Celebrating the end of the project.}
\end{milestones}

%%% Local Variables:
%%% mode: latex
%%% TeX-master: "proposal"
%%% End:



% ---------------------------------------------------------------------------
% Include Work package descriptions
% ---------------------------------------------------------------------------

\newpage
\subsubsection{Work Package Descriptions}\label{sec:workpackages}
%% WP titles and order are defined in deliverables.tex
%%% work package style may be broken -- fix this!!

%% Local WP number counter - should possibly be global and hidden?
\begin{workplan}
\begin{workpackage}[id=WP1,wphases=0-48,
  short=First WP,% for Figure 5.
  title=First Work Package,
  lead=PAR1,
  PAR1RM=12,
  PAR2RM=6,
  PAR3RM=24]

\begin{wpobjectives}
  There are objectives.
  \begin{compactitem}
  \item Item 1.
  \item Item 2.
  \end{compactitem}
\end{wpobjectives}

\begin{wpdescription}
  Description of WP1 work.
\end{wpdescription}

\begin{tasklist}

  \begin{task}[title=TASK1,id=task1,PM=15,lead=PAR1,wphases=0-30!0.5]

    First task of the project.
    
  \end{task}

  \begin{task}[title=TASK2,id=task2,PM=15,lead=PAR2,wphases=12-42!0.5]

    Second task of the project.
    
  \end{task}

\end{tasklist}

\begin{wpdelivs}
  \begin{wpdeliv}[due=12,id=mydeliv1,dissem=PU,nature=DEM,lead=PAR1]
      {First deliverable, after 1 year.}
  \end{wpdeliv}
  \begin{wpdeliv}[due=24,id=mydeliv2,dissem=PU,nature=DEM,lead=PAR2]
      {Second deliverable, after 2 years.}
\end{wpdeliv}
\end{wpdelivs}

\end{workpackage}

\begin{workpackage}[id=management,type=MGT,wphases=0-48!.2,swsites,
  title=Project Management,short=Management,
  lead=PAR1,PAR1RM=12,PAR2RM=6,PAR3RM=6]

\begin{wpobjectives}
  Description of this WP's objectives.
\end{wpobjectives}

\begin{wpdescription}
  Management of the project.
\end{wpdescription}

\begin{tasklist}
\begin{task}[title=Management tasks 1,id=mgt-task-1,lead=PAR1,PM=18,wphases={0-48!0.25,0-24!0.25},
  partners={PAR1,PAR2}]

  Management tasks 1.

\end{task}

\begin{task}[title=Management tasks 2,id=mgt-task-1,lead=PAR3,PM=6,wphases=24-48!0.25]

  Management tasks 2.

\end{task}

\begin{wpdelivs}
  \begin{wpdeliv}[due=1,miles=start,id=ca,dissem=CO,nature=R,lead=PAR1]{Consortium Agreement}
  \end{wpdeliv}
  \begin{wpdeliv}[due=1,miles=start,id=tickets,dissem=PU,nature=DEC,lead=PAR1]{Infrastructure}
  \end{wpdeliv}
  \begin{wpdeliv}[due=6,miles=start,id=data,dissem=PU,nature=R,lead=PAR3]{Hosting}
  \end{wpdeliv}
\end{wpdelivs}

\end{workpackage}

%%% Local Variables: 
%%% mode: latex
%%% TeX-master: "../proposal"
%%% End: 

\end{workplan}

%%% Local Variables:
%%% mode: latex
%%% TeX-master: "../proposal"
%%% End:


\newpage
\subsection{Management Structure and Procedures}
\label{sect:mgt}

The project is managed.

\milestonetable

%%% Local Variables:
%%% mode: latex
%%% TeX-master: "proposal"
%%% End:


\draftpage
\subsection{Consortium as a Whole}
\input{consortium.tex}
\draftpage

\subsection{Resources to be Committed}
\eucommentary{Please provide the following:
\begin{compactitem}
\item
a table showing number of person/months required (table 3.4a)
\item
a table showing 'other direct costs' (table 3.4b) for participants where
those costs exceed 15\% of the personnel costs (according to the budget
table in section 3 of the administrative proposal forms)
\end{compactitem}}

\subsubsection{Management Level Description of Resources and Budget}
\label{sect:budget-details}

\paragraph{Staff efforts}

\eucommentary{Please indicate the number of person/months over the whole
duration of the planned work, for each work package, for each participant.
Identify the work-package leader for each WP by showing the relevant
person-month figure in bold.}

\wpfig[label=fig:staffeffort,caption=Summary of Staff Efforts]

\subsubsection{Resource summaries for consortium member sites}
\label{resources.summary}

%%%%%%%%%%%%%%%%%%%%%%%%%%%%%%%%%%%%%%%%%%%%%%%%%%%%%%%%%%%%%%%%
%
% Guidelines for completion of partner specific resource summary:
%
%
% Please explain how many person months for each person are
% requested. Say who is the local lead. Say anything that helps to
% understand why people are recruited as you plan, in particular if
% this deviates from having one research for 48 months.  We can also
% use this bit of the proposal (and the table, see below) to address
% any other unusual arrangements.
%
%
% The table should contain all non-staff costs (the EU requests that
% this table must be present if the non-staff costs exceed
% 15% of the total cost, but it is good practice and will show
% openness and transparency that we show the data for all partners).
%
% Link back from the table to the work packages and tasks for which
% the expenses are required. Add information that makes it easier to
% understand why the expenses are justified.
%
%     To refer to a task in a work package, use "\taskref{WP-ID}{TASK-ID}" where
%     WP-ID is the ID of the work package:
%        WP#: WP-ID - full title
%        ----------------------
%        WP1: 'management' - Management
%        WP2: 'community' - Community Building and Engagement
%        WP3: 'component-architecture' - Component Architecture
%        WP4: 'UI' - User interfaces
%        WP5: 'hpc' - High Performance Computing
%        WP6: 'dksbases' - Data/Knowledge/Software-Bases
%        WP7: 'social-aspects' - Social Aspects
%        WP8: 'dissem' - Dissemination
%
%
%     and "TASK-ID" is the ID of the task. You can set this using
%
%       \begin{task}[id=TASK-ID,title=Math Search Engine,lead=JU,PM=10,lead=JU]
%
%     To refer to deliverables, use "\delivref{WP-ID}{DELIV-ID}" where DELIV-ID is
%     the ID of the deliverable that can be set like this:
%
%       \begin{wpdeliv}[due=36,id=DELIV-ID,dissem=PU,nature=DEM]
%           {Exploratory support for semantic-aware interactive widgets providing views on objects
%           represented and or in databases}
%       \end{wpdeliv}
%
%
% The table is pre-populated with entries most sites are likely
% to need. If a line does not apply to you, just delete it. If you need
% an extra line, then add it. Use common sense: the number of rows should not
% be very big, but at the same time it is useful to give some breakdown/explanation
% of costs.
%
%
% Eventually, try to create you entry similar in style to the others.
% (The Southampton entry is fully populated, so use this as guidance
% if in doubt.)
%
%
%%%%%%%%%%%%%%%%%%%%%%%%%%%%%%%%%%%%%%%%%%%%%%%%%%%%%%%%%%%%%%%%

%%%%%%%%%%%%%%%%%%%%%%%%%%%%%%%%%%%%%%%%%%%%%%%%%%%%%%%%%%%%%%%%%%%%%%%%%%%%%%
\paragraph{Resources PAR1}

PAR1 will consist of PAR1P1 and PAR1P2.

\paragraph{Resources PAR2}

PAR2 will consist of PAR2P1.

\paragraph{Resources PAR3}

PAR1 will consist of PAR3P1 and PAR3P2.

%%% Local Variables:
%%% mode: latex
%%% TeX-master: "proposal"
%%% End:


% ---------------------------------------------------------------------------
%  Section 4: Members of the Consortium
% ---------------------------------------------------------------------------

\newpage

\eucommentary{This section is not covered by the page limit.\\
The information provided here will be used to judge the operational capacity.}

\section{Members of the Consortium}

\subsection{Participants}

\eucommentary{Please provide, for each participant, the following (if available):\\
\begin{compactitem}
\item
a description of the legal entity and its main tasks,
with an explanation of how its profile matches the tasks in the proposal;
\item
a curriculum vitae or description of the profile of the persons,
including their gender, who will be primarily responsible for carrying
out the proposed research and/or innovation activities;
%
this includes a description of the profile of the to-be-recruited personnel
\item
a list of up to 5 relevant publications, and/or products, services
(including widely-used datasets or software), or other achievements
relevant to the call content;
\item
a list of up to 5 relevant previous projects or activities, connected
to the subject of this proposal;
\item
a description of any significant infrastructure and/or any major items
of technical equipment, relevant to the proposed work;
\item
any other supporting documents specified in the work programme for this call.
\end{compactitem}}

\begin{sitedescription}{PAR1} \label{desc:PAR1}

The PAR1 is very good.

\subsubsection*{Curriculum vitae of the investigators}

\begin{participant}[type=PI,PM=12,gender=female,salary=5500]{Person One}

  Person One is an experienced professor and will lead the project.

\end{participant}

%%% Local Variables:
%%% mode: latex
%%% TeX-master: "../proposal"
%%% End:

\begin{participant}[type=R,PM=48,gender=male,salary=5500]{Person Two}

  Person Two is an experienced researcher and will participate full time to the
  project.

\end{participant}

%%% Local Variables:
%%% mode: latex
%%% TeX-master: "../proposal"
%%% End:


\begin{participant}[type=res,PM=48,salary=5500]{NN}
\end{participant}
\begin{participant}[type=res,PM=36,salary=5500]{NN}

We need researchers. Two, for instance.

\end{participant}

\begin{participant}[type=res,PM=24,salary=3932]{NN}
  We will hire an experienced part time project manager to help with
  the overall management during the whole duration of \TheProject.
\end{participant}

\subsubsection*{Publications, achievements}

\begin{compactenum}
\item Leadership.
\item Coauthoring.
\end{compactenum}


\subsubsection*{Previous projects or activities}

\begin{compactenum}
\item Hosting.
\item Co-organising.
\end{compactenum}

\subsubsection*{Significant infrastructure}

We have building, at PAR1.

\end{sitedescription}



\begin{draft}
\vspace{1cm}\TOWRITE{PAR1P1}{Complete check list below -- delete completed items if you wish}

\begin{verbatim}
- [ ] checked that sum of person months put into finance request is
  the same as sum of person months associated with the Work Packages
  (in proposal.tex, as defined as part of the \begin{workpackage}"
  command.
  
- [ ] completed site specific resource summary in resources.tex,
  including table of non-staff costs.

\end{verbatim}
\end{draft}

%%% Local Variables: 
%%% mode: latex
%%% TeX-master: "../proposal"
%%% End: 

\clearpage
\input{Participants/PAR2.tex}
\clearpage
\input{Participants/PAR3.tex}
\clearpage

\subsection{Third Parties Involved in the Project (including use of third party resources)}
\label{section:ThirdParties}

\paragraph{Third Party 1}\ 

\eucommentary{Please complete, for each participant, the table
(see page 27 of "VRETemplate.PDF"),
or simply state "No third parties involved", if applicable.}

Third Party 1 (hereafter TP1) will work on the project.

\paragraph{Other participants}\ 

For other participants, the only subcontracting costs will be for audit.

\bgroup
\def\arraystretch{1.5}  % 1 is the default
\noindent \begin{tabular}{|p{0.6\textwidth}|c|}
\hline
Does the participant plan to subcontract certain
tasks & Yes \\
\hline
\multicolumn{2}{|l|}{Audit} \\
\hline
Does the participant envisage that part of its work
is performed by linked third parties & No \\
\hline
\multicolumn{2}{|l|}{} \\
\hline
Does the participant envisage the use of
contributions in kind provided by
third parties & No \\
\hline
\multicolumn{2}{|l|}{} \\
\hline
\end{tabular}
\egroup

%No third parties involved.

% ---------------------------------------------------------------------------
%  Section 5: Ethics and Security
% ---------------------------------------------------------------------------

\newpage

\section{Ethics and Security}

\eucommentary{This section is not covered by the page limit.}

\subsection{Ethics}

\eucommentary{
If you have entered any ethics issues in the ethical issue table in the administrative proposal forms, you must:\\
$\bullet$ submit an ethics self-assessment, which: \\
-- describes how the proposal meets the national legal and ethical requirements of the
country or countries where the tasks raising ethical issues are to be carried out;\\
-- explains in detail how you intend to address the issues in the ethical issues table, in
particular as regards:
research objectives (e.g. study of vulnerable populations, dual use, etc.),
research methodology (e.g. clinical trials, involvement of children and related
consent procedures, protection of any data collected, etc.),
the potential impact of the research (e.g. dual use issues, environmental damage,
stigmatisation of particular social groups, political or financial retaliation,
benefit-sharing, malevolent use , etc.)\\
$\bullet$ provide the documents that you need under national law (if you already have them), e.g.:\\
-- an ethics committee opinion;\\
-- the document notifying activities raising ethical issues or authorizing such activities\\
If these documents are not in English, you must also submit an English summary of them
(containing, if available, the conclusions of the committee or authority concerned).\\
If you plan to request these documents specifically for the project
you are proposing, your request must contain an explicit reference to the project title}

\subsection{Security}

Please indicate if your proposal will involve:

\begin{compactitem}
\item activities or results raising security issues: NO
\item 'EU-classified information' as background or results: NO
\end{compactitem}
\end{proposal}
\TOWRITE{ALL}{Search through final.pdf ('make final') and look for questions marks ?? and XX and YY and XYZ as place holders where people intended to later add a link, or where a link is broken.}
\end{document}

%%% Local Variables:
%%% mode: latex
%%% TeX-master: t
%%% End:

